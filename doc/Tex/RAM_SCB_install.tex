Installation and compilation of RAM-SCB requires a Fortran compiler, MPI 
compiled against the chosen Fortran compiler, a Perl interpreter, and several 
external libraries which have additional requirements (most notably a C 
compiler, typically standard on Unix-like systems.)  The configuration is 
done with the Config.pl script; compilation with GNU Make.  The Config/Make 
system follows the tiered makefile standards of the Space Weather Modeling 
Framework (SWMF) project; users of the SWMF should feel right at home using
RAM-SCB.

\section{Installation of Required Libraries}

There are three required libraries that must be installed to use RAM-SCB: 
\href{http://ngwww.ucar.edu/}{NCAR Graphics}, 
\href{http://www.unidata.ucar.edu/software/netcdf/}{Unidata's NetCDF library} 
and \href{http://w3.pppl.gov/ntcc/PSPLINE/}{NTCC's Princeton Spline (Pspline) 
library}. It is possible to find pre-compiled binaries for each library, 
allowing the user to skip the configure, make, and install steps.  However, 
if a binary cannot be found that matches your system and compilers, you will 
be forced to install from source.  Some systems may already have these 
libraries installed; be sure to check before going through unneccessary 
installation work.

\subsection{Installation From Source}
 
Tarballs of the source code can be obtained through these URLs:
\begin{enumerate}
\item{\href{http://www.ncl.ucar.edu/Download/}{http://www.ncl.ucar.edu/Download/} (login required)}

\item{\href{http://w3.pppl.gov/rib/repositories/NTCC/files/pspline.tar.gz}{http://w3.pppl.gov/rib/repositories/NTCC/files/pspline.tar.gz}}

\item{\href{http://www.unidata.ucar.edu/downloads/netcdf/ftp/netcdf-4.0.1.tar.gz}{http://www.unidata.ucar.edu/downloads/netcdf/ftp/netcdf-4.0.1.tar.gz}}
\end{enumerate}


Unpack the tarballs in the usual fashion:
\begin{verbatim}
tar -xzvf netcdf-4.0.1.tar.gz
\end{verbatim}

Because these libraries are not owned or maintained by the authors of RAM-SCB, 
only minimal installation instructions are provided here.  The reader is 
encouraged to explore the websites and help files associated with each package.


\subsubsection{Ncarg Quick Install}
Installation of the NCAR Graphics library from source is an involved process.  
It is strongly suggested that the user first check their system to see if the 
library is already installed or install via binary.

\subsubsection{NetCDF Quick Install}
Unpack the NetCDF tarball, {\tt cd} into the unzipped directory, and use the 
following commands:
\begin{verbatim}
./configure --prefix=/Users/ram_user/netcdf_lib
make install
\end{verbatim}
This will install the NetCDF library in {\tt /Users/ram\_user/netcdf\_lib} 
(the home directory for user \textit{ram\_user}); change the value of 
{\tt --prefix} to select the installation destination.  For installation, 
NetCDF requires only a C compiler.  {\tt gcc} is strongly suggested.

\subsubsection{Pspline Quick Install}
Pspline requires NetCDF, so be sure that NetCDF is correctly installed before 
continuing.  After unpacking the tarball, compile the library using
\begin{verbatim}
make FORTRAN_VARIANT=Portland NETCDF_DIR=~/netcdf_lib
\end{verbatim}
It is important to specify the Fortran compiler that coincides with the 
compiler you will use to make the RAM-SCB executable.  Other options accepted 
by the {\tt FORTRAN\_VARIANT} variable are {\tt PathScale},  {\tt GCC}(gfortran
and g95), {\tt NagWare},  {\tt Fujitsu}, {\tt Intel}, and {\tt Absoft}.  
Ensure that the value of {\tt NETCDF\_DIR} coincides with the installation 
directory of NetCDF.

{\tt make} may fail during the creation of the Pspline test utilities.  If 
this happens, check for the creation of a {\tt lib} directory and several 
\verb*l*.al files contained within.  If they exist, disregard the error 
message and continue to the next step.

Install Pspline as follows (changing the value of {\tt PREFIX} to select 
installation directory):
\begin{verbatim}
make install PREFIX=~/psline_portland
\end{verbatim}
On machines where you will use different compilers, it is helpful to add the 
compiler name to the name of the installation directory.

\subsection{Libraries From Modules}
Well maintained clusters often use the {\tt module} interface for loading
and unloading software packages.  Be sure to explore the available packages
on a new system before going through the arduous task of installing from source!

\begin{verbatim}
module avail
\end{verbatim}
\noindent
will list all available software libraries on the system.  You can load a 
library, unload a library, and list loaded software using the module commands:

\begin{verbatim}
module load package_name
module unload package_name
module list
\end{verbatim}

As an example, here is what you would type on NASA's Pleiades cluster to get
started with RAM-SCB:

\begin{verbatim}
module load comp/intel/10.1.021_64
module load mpi/mpt.1.25
module load netcdf/4.0-i10.1
module load ncarg/4.4.2/intel
module load python/2.6.1
\end{verbatim}

Intel Fortran, MPI, NetCDF and NCAR Graphics are now loaded and ready to use.
Place such commands your shell configuration 
script ({\tt \verb*l~l/.cshrc} or {\tt \verb*l~l/.bashrc} depending on your 
shell) to load modules upon login.  Do not forget to point RAM-SCB to these
libraries as listed below.

\section{Pointing RAM-SCB to External Libraries}
RAM-SCB must be explicitly pointed to libraries installed in non-standard 
places.
There are several ways for RAM-SCB to find the required external libaries, 
summarized in Table \ref{tab:libs}.  The most permanant and convenient method 
is to set an environment variable that RAM-SCB will search for upon 
installation.  They can be set with the following commands:

Bash Shell: {\tt export NETCDFDIR=}\textit{installation path}

C Shell: {\tt setenv NETCDFDIR ``}\textit{installation path}\verb*l''l

Placing these commands into configuration files (e.g. {\tt \verb*l~l/.bashrc} 
or {\tt \verb*l~l/.cshrc}) will load the environment variables at login.  This 
is preferable to typing the commands every session.  Note that the installation
path listed should point to the top level directory for the installation.  
For example, {\tt /Users/ram\_user/ncarg} is sufficient but 
{\tt /Users/ram\_user/ncarg/lib} will cause an error.

Alternatively, it is possible to set the install paths for the external 
libraries using Config.pl during installation.  The use of this method is 
detailed in the next section.  Even if you have set the environment variables, 
using the Config.pl switches will override them.  These switches are 
\emph{only} recognized in conjunction with the {\tt -install} switch 
of Config.pl.

Finally, if you set neither the environment variables or the Config.pl 
switches, the Fortran compiler will look for the libaries in the default 
directories.  This may be useful to test if the libraries are already 
properly installed by the system administrator.  

\emph{If you change the library locations by any method, 
you must reinstall the code.} 

\begin{table}[ht]
  \centering
  \begin{tabular}{l l l}
  \hline\hline
  Library & Environment Variable & Config.pl Switch\\
  \hline
  NCAR Graphics & {\tt NCARGDIR=[...]} & {\tt -ncarg=[...]}\\
  NetCDF & {\tt NETCDFDIR=[...]} & {\tt -netcdf=[...]}\\
  Pspline & {\tt PSPLINEDIR=[...]} & {\tt -psline=[...]}\\
  \end{tabular}
\caption{List of required libraries and methods for expressing their location 
  to RAM-SCB.  Note that the Config.pl switches override environment variables.
  If none are given, the Fortran compiler will search in the default library 
  location.}
\label{tab:libs}
\end{table}

\section{Installation, Configuration, and Compiling \label{subchap:install}}
{\tt Config.pl} handles the installation and configuration of RAM-SCB.  
To view the installation and configuration status, or to view help, use 
the following commands:
\begin{verbatim}
Config.pl
Config.pl -h
\end{verbatim}

To install the code, use the {\tt -install} flag:

\begin{verbatim}
Config.pl -install
\end{verbatim}


Although RAM-SCB will try to use reasonable defaults based on your system, 
there are a number of flags that allow you to customize your installation:
\begin{itemize}
\item{Use {\tt -compiler} to select the Fortran 90 compiler.  
  Common choices include pgf90, ifort, gfortran, and f95(for NAG).}
\item{The {\tt -mpi} flag allows the user to pick which version of MPI to use.
  Choose from mpich, mpich2, Altix, and openmpi.  Alternatively, the 
  {\tt -nompi} flag may be set with no value.  This option is fragile.}
\item{Set {\tt -ncdf} to the path of the NetCDF library installation.  
  If used, this flag overrides the environment variable NETCDFDIR.}
\item{Set {\tt -pspline} to the path of the Pspline library installation.  
  If used, this flag overrides the environment variable PSPLINEDIR.}
\end{itemize}

To exemplify a typical installation, imagine a machine with several different 
Fortran compilers available.  For each compiler available, the user has a 
corresponding installation of Pspline.  The user will use this command to 
properly install RAM-SCB:
\begin{verbatim}
Config.pl -install -compiler=pgf90 -mpi=mpich2 -pspline=~/libs/pspline_portland/
\end{verbatim}

There are other Config.pl options that set up real precision, debug flags, 
and optimization level.  Use {\tt Config.pl -h} to learn about the available 
options.

After the code has been properly configured, compiliation is simple:
\begin{verbatim}
make
\end{verbatim}

Compilation is most likely to fail for two reasons.  The first is RAM-SCB
not finding MPI or another key library.  The second is MPI or an external
library that is installed using a mix of different Fortran compilers.  Be
vigilant when installing each library!

To remove object files before a fresh compilation, use
\begin{verbatim}
make clean
\end{verbatim}

To uninstall RAM-SCB, simply use
\begin{verbatim}
Config.pl -uninstall
\end{verbatim}

\section{Testing the Installation}
Running the RAM-SCB tests is an excellent way to evaluate the success and 
stability of your installation.  To run the tests, simply type
\begin{verbatim}
make test
\end{verbatim}
This will compile RAM-SCB, create a run directory, perform a short simulation 
and compare the results to a reference solution.  If there is a significant 
difference between the test and reference solution, the test will fail.  
Details can be found in \verb*Q*.testQ files in the RAM-SCB directory.  
For a full description of running and interpreting test results, see Chapter 
\ref{chp:test}.

\section{Building Documentation}
To generate a PDF of the latest User Manual, type
\begin{verbatim}
make PDF
\end{verbatim}
The document will be located in the \verb*ldoc/l directory.
