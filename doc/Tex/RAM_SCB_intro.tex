The \textbf{R}ing current \textbf{A}tmosphere interactions \textbf{M}odel with 
\textbf{S}elf \textbf{C}onsistent magnetic field (\textbf{B}) is a unique code 
that combines a kinetic model of ring current plasma with a three dimensional 
force-balanced model of the terrestrial magnetic field.  The kinetic portion, 
RAM, solves the kinetic equation to yield the bounce-averaged distribution 
function as a function of azimuth, radial distance, energy and pitch angle 
for three ion species ($H^{+}$, $He^{+}$, and $O^{+}$) and, optionally, 
electrons.  The domain is a circle in the Solar-Magnetic (SM) equatorial plane 
with a radial span of 2 to 6.5 $R_{E}$.  It has an energy range of 
approximately 100\,$eV$ to 500\,$KeV$.  The 3-D force balanced magnetic field 
model, SCB, balances the $\textbf{J} \times \textbf{B}$ force with the 
divergence of the general pressure tensor to calculate the magnetic field 
configuration within its domain.  The domain ranges from near the Earth's 
surface, where the field is assumed dipolar, to the shell created by field 
lines passing through the SM equatorial plane at a radial distance of 6.5 
$R_{E}$.  The two codes work in tandem, with RAM providing anisotropic pressure
to SCB and SCB returning the self-consistent magnetic field through which RAM 
plasma is advected.

RAM-SCB has grown from a research-grade code with limited options and static 
magnetic field (RAM) to a rich, highly configurable research and operations 
tool with a multitude of new physics and output products.  This manual provides
a guide to users who want to learn how to install, configure, and execute 
RAM-SCB simulations.  While the code is designed to make these steps as
straight-forward as possible, it is strongly recommended that users review
the publications listed in the Bibliography to ensure a thorough understanding 
of the physics included in the model.  Additionally, all users are asked to 
review the terms of use.

\section{About This Manual}
Users who want to install and begin quickly should start at Chapter 
\ref{chp:quick}, which quickly outlines the path from installation to 
simulation with little detail.  The installation process is discussed fully 
in Chapter \ref{chp:install}.  Instructions on performing simulations, 
as well as several example simulations, are given in Chapter \ref{chp:run}. 
 An outline of using this code in the Space Weather Modeling Framework is 
found in Chapter \ref{chp:swmf}. Useful scripts included in the distribution
are described, in brief, in Chapter \ref{chp:scripts}.
Finally, a complete list of all param file commands is found in Chapter \ref{chp:param}.

\section{TERMS OF USE \& DISTRIBUTION POLICY}

Use of the RAM-SCB software implies agreement with the terms herein.
RAM-SCB is open source software that has been developed at Los Alamos National Laboratory (LANL). The code is based on physics and
numerical methods detailed in the following publications and references therein:

1. Jordanova, V. K. et al. (2006), Kinetic simulations of ring current 
evolution during the Geospace Environment Modeling challenge events, 
J. Geophys, Res., 111, A11S10, doi:10.1029/2006JA011644. \\

2. Zaharia, S. et al. (2006), Self-consistent modeling of magnetic fields 
and plasmas in the inner magnetosphere: Application to the geomagnetic storm,
J. Geophys. Res., 111, A11S14, doi:10.1029/2006JA011619. \\

Redistribution and use in source and binary forms, with or without modification, are permitted provided that the following conditions are met: \\

1. Redistributions of source code must retain the copyright notice, this list of conditions and the following disclaimer. \\

2. Redistributions in binary form must reproduce the copyright notice, this list of conditions and the following disclaimer in the documentation and/or other materials provided with the distribution. \\

3. Neither the name of Los Alamos National Security, LLC, Los Alamos National Laboratory, LANL, the U.S. Government, nor the names of its contributors may be used to endorse or promote products derived from this software without specific prior written permission. \\

This software has been authored by an employee or employees of Los Alamos National Security, LLC,
operator of the Los Alamos National Laboratory (LANL) under Contract No. DE-AC52-06NA25396 with the
U.S. Department of Energy. This software is provided by the copyright holders and contributors "as is" and
any express or implied warranties, including, but not limited to, the implied warranties of merchantability
and fitness for a particular purpose are disclaimed. In no event shall the copyright holder or contributors be
liable for any direct, indirect, incidental, special, exemplary, or consequential damages (including, but not
limited to, procurement of substitute goods or services; loss of use, data, or profits; or business interruption)
however caused and on any theory of liability, whether in contract, strict liability, or tort (including negligence
or otherwise) arising in any way out of the use of this software, even if advised of the possibility of such damage. \\

The references below represent critical development milestones for RAM-SCB. Please consider citing
these works to give the developers proper credit.


\begin{table}[ht]
  \centering
  \begin{tabular}{l|l}
    Citation & Information \\
    \hline
    \hline
    Jordanova et al. [1996] & First description of ring current model (RAM) using dipolar magnetic field \\
    \hline
    Jordanova et al. [2006] & First extension of RAM for non-dipolar magnetic field and coupling with SCB \\
    \hline
    Zaharia et al. [2006] & Description of SCB model and coupling with RAM \\
    \hline
    Jordanova et al. [2010] & Full description of RAM extension for non-dipolar magnetic field \\
    \hline
    Welling et al. [2011] & First full description of one-way coupling with SWMF \\
    \hline
    Welling et al. [2015] & Description of two-way coupling of RAM-SCB
  with SWMF \\
    \hline
  \end{tabular}
\end{table}




